\chapter{Competencias desarrolladas}

Una competencia nos permite identificar, seleccionar, coordinar y movilizar de manera articulada e interrelacionada un conjunto de saberes diversos en el marco académico. Éstas enfatizan lo que el estudiante o el egresado es capaz de hacer al término de su proceso formativo y en las estrategias que le permiten aprender de manera autónoma en el contexto académico o laboral.

\section{Competencias desarrolladas o aplicadas}

En el listado que se muestra a continuación se presentan las competencias desarrolladas o aplicadas durante la estancia en Limón Almacenes S.A. de C.V.

\begin{itemize}
	\item Conocer, comprender y aplicar las estructuras de datos, métodos de ordenamiento y búsqueda para la optimización del rendimiento de soluciones de problemas del contexto.
	\item Desarrollo de soluciones de software utilizando programación concurrente, programación de eventos, que soporte interfaz grafica e incluya dispositivos móviles.
	\item Diseño e implementación de objetos de programación que permitan resolver situaciones reales y de ingeniería.
	\item Identificación de las implicaciones actuales en la
	programación móvil.
	\item Identificación de características de los diferentes emuladores para dispositivos móviles.
	\item Aplicación de un lenguaje para la solución de problemas para dispositivos móviles.
	\item Análisis de requerimientos y diseño de bases de datos para generar soluciones al tratamiento de información basándose en modelos y estándares.
	\item Implementación de bases de datos para apoyar la toma de decisiones considerando las reglas de negocio.
	\item Instalación, configuración y administración de un gestor de base de datos para el manejo de la información de una organización, optimizando la infraestructura computacional existente.
	\item Configuración, implementación y administración de redes que permitan resolver situaciones reales y de ingeniería.
	\item Conocer los principios lógicos y funcionales de la programación para aplicarlos en la resolución de problemas.
\end{itemize}
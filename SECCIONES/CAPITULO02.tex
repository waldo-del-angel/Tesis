\chapter{Marco teórico}
	En el capitulo dos se presenta los trabajos previos que abordan de manera directa o indirecta temas relacionados al desarrollo del software.
\section{Definiciones clave del estudio}
	Se describe de manera breve cada una las herramientas tecnológicas que contribuyen a la elaboración del proyecto.
	\subsection{Herramientas administrativas}
	Las herramientas administrativas son utilizadas en la empresa Limón Almacenes S.A. de C.V. para la administración de sus sistemas de software y manejo de bases de datos.
	\subsubsection{Techni-web}
	La empresa desarrolladora del software \textcite{Techniweb} la describe como ``una aplicación que está diseñada para resolver de manera rápida y sencilla la gestión de facturación (compras y ventas), stock y contabilidad de una empresa" y que ``está preparada para funcionar tanto en pequeñas como en grandes empresas con redes distribuidas con gran cantidad de usuarios".
	\subsubsection{Firebird}
	\textcite{Firebird} señala que "Firebird es un administrador de bases de datos relacionales que se deriva del código fuente de InterBase 6.0, de Borland. Es de código abierto y no tiene licencias duales".
	\subsubsection{SQL Explorer}
	Es un entorno de desarrollo para trabajar con Firebird SQL. Gestión Techni-web proporciona esta herramienta de forma nativa.
	\subsubsection{EMS Manager for Interbase \& Firebird}
	De acuerdo con la documentación en línea de \textcite{Solutions2016} es ``una poderosa herramienta para Sistemas de Administración de Base de Datos en Interbase/Firebird". Es un entorno de trabajo más completo para el manejo de las bases de datos realizadas en Firebird.
	\subsection{Herramientas de desarrollo}
	Las herramientas de desarrollo son utilizadas son utilizadas por el programador para la elaboración del software.
	\subsubsection{Android Studio}
	Como expresa la guía de usuario en línea de \textcite{Android}, es el entorno de desarrollo integrado (IDE) oficial para el desarrollo de aplicaciones para Android y se basa en IntelliJ IDEA.
	\subsubsection{Kotlin}
	Kotlin es un proyecto gratuito y de código abierto registrado bajo la licencia de Apache 2.0. Su desarrollo y distribución como software gratuito están asegurados por Kotlin Foundation. La plataforma de aprendizaje en línea de habla hispana \textcite{Platzi} señala que "Kotlin toma lo mejor y quita lo que le sobra a Java para enfocarse en la productividad".
	\subsubsection{Jaybird}
	Con base en la documentación oficial de Jaybird de \textcite{Jaybird}: "Jaybird es un conjunto de controladores JCA/JDBC para conectarse al servidor de base de datos Firebird". Es el driver oficial para las conexiones entre aplicaciones Java y Firebird.
	\subsubsection{Zxing}
	Es una biblioteca de procesamiento de imágenes de código de barras implementada en Java (realizada por \citeauthor{Zxing}) con puertos a otros lenguajes. Tiene soporte para productos 1D, 1D industrial y códigos de barras 2D.
	\subsection{Herramientas de diseño y maquetado}
	Las herramientas de desarrollo son utilizadas por el programador para la elaboración de diseños y maquetados correspondientes al desarrollo del software.
	\subsubsection{Balsamiq Mockups 3}
	Es una herramienta rápida de baja fidelidad para la interfaz de usuario que reproduce la experiencia de dibujar en un bloc de notas o pizarra, pero utilizando una computadora \parencite{Basamiq}.
	\subsubsection{Adobe XD CC}
	En la página oficial de \textcite{Adobe} se describe a Adobe XD como "la forma más rápida de diseñar, crear prototipos y compartir cualquier experiencia de usuario, desde sitios web y aplicaciones móviles hasta interacciones por voz y mucho más".
	\subsubsection{Dia}
	Dia es una aplicación para crear diagramas técnicos. Su interfaz y sus características tienen un patrón holgado según el programa de Windows Visio \parencite{Dia}.

\section{Estudios previos}
	En esta sección se enlistan los productos en el mercado los cuales persiguen un objetivo similar al proyecto en cuestión.
	\subsection{Software}
		Se presenta el software presente en el mercado los cuales tienen una orientación similar al proyecto en cuestión.
		\subsubsection{RadarPrice}
		Aplicación móvil para plataformas iOS y Android: Busca y localiza cualquier producto al mejor precio entre miles de tiendas online y físicas, escaneando su código de barras. Este sofware es propiedad de ONYOUGO DREAMS S.L. en Barcelona, España \parencite{RadarPrice}.
		\subsubsection{SCANPET}
		Aplicación móvil en Android la cual escanea y lee códigos de barras buscando en un archivo Excel, que actúa como base de datos. Utiliza la cámara de fotos del dispositivo móvil para leer los códigos de barras \parencite{Domus}.
		
		Entre sus funcionalidades se encuentran: inventario de almacén, control de stock, terminal WiFi.
	\subsection{Hardware}
		Ciertas compañías desarrollan hardware específico para sus aplicaciones que integran un lector de código de barras, tal es el caso de MaxMovil.
		\subsubsection{Tigersan PDA 4G}
		Es un PDA desarrollado por la empresa MaxMovil orientada al control de inventarios, con Safe Droid como sistema operativo. EL dispositivo cuenta con un escáner de coódigo de barras 1D integrado \parencite{MaxMovil}.